The algorithm behaves quite similarly for the different problems we have tested it on. However, as expected, some starting points yields weird results such as the algorithm taking zero-length steps and so on. When selecting too large or small $\mu/\epsilon$ the algorithm does not converge at all.  
If the solution does not have a minimum there is a risk that the solver gets caught in an infinite loop. This can be solved by keeping track of the number of iterations performed in the line search or simply testing the function value in the limit to infinity, however we did not implement this. 
