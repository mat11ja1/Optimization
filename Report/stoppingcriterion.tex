In the line search we chose the stopping criterion which had two aspects. Firstly, the difference between the functional values on the endpoints of the interval need to be within the tolerance range. Secondly, the difference between the center of the interval and the average value of the endpoints needs to be within the interval. See below for the stopping function
$$ while\; abs(f(a)-f(b)) > tol\; \&\&\;abs(f((a+b)/2)-(f(a)+f(b))/2) > tol$$

The stopping criterion used in the outer loop is instead to stop when both the difference between the found points are within the tolerance region or the difference between the functional values for subsequent points is small enough. We chose to do use both of the two differences since if we only look at the functional value we might have a problem with symmetrical functions. For the problems we have implemented we can only see a huge difference for the problem where we minimize $\min e^{x_1 x_2 x_3 x_4} $, where there is a factor of 10 times more iterations if both conditions needs to be met. For the other problems there is at most a difference of a couple iterations.