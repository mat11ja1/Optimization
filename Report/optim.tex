\documentclass{article}

\usepackage[version=3]{mhchem} % Package for chemical equation typesetting
\usepackage{siunitx} % Provides the \SI{}{} and \si{} command for typesetting SI units
\usepackage{graphicx} % Required for the inclusion of images
\usepackage{natbib} % Required to change bibliography style to APA
\usepackage{amsmath} % Required for some math elements 
%\usepackage[]{mcode}
\setlength\parindent{0pt} % Removes all indentation from paragraphs
\renewcommand{\labelenumi}{\alph{enumi}.} % Make numbering in the enumerate environment by letter rather than number (e.g. section 6)
\usepackage[margin=1in]{geometry}
\usepackage{mathtools}
\usepackage{epstopdf}
\usepackage{times} % Uncomment to use the Times New Roman font
%----------------------------------------------------------------------------------------
%	DOCUMENT INFORMATION
%---------------------------------------------------------------------------------------
\begin{document}

\title{Optimization, Quasi-Newton methods\\  FMA051} % Title

\author{Jonathan Andersson \\ Adam Jalkemo \\ Emil Westenius} % Author name

\date{\today} % Date for the report


\maketitle % Insert the title, author and date
\pagebreak


% If you wish to include an abstract, uncomment the lines below
% \begin{abstract}
% Abstract text
% \end{abstract}
%----------------------------------------------------------------------------------------
%	SECTION 1
%----------------------------------------------------------------------------------------
\section{Code discussion}
\subsection{Line search}
\subsection{Stop conditions}
%B�de DFP, BFGS & linesearch



%----------------------------------------------------------------------------------------
%	SECTION 2
%----------------------------------------------------------------------------------------
\section{Problems}
\subsection{Rosenbrock}
\subsubsection{Optimal points and function value}
\subsection{$\min e^{x_1 x_2 x_3 x_4}$ subject to $x_1^2 + x_2^2 + x_3^2 + x_4^2 + x_5^2 = 10$, $x_2 x_3 = 5 x_4 x_5 $, $x_1^3 + x_3^3 = -1$}
\subsubsection{Optimal points and function value}
\subsection{Exercise 9.3}
The problem formulation is
\begin{equation*}
\begin{aligned}
& \underset{\mathbf{x}}{\text{minimize}}
& & e^{x_1} + x_1^2 + x_1 x_2 \\
& \text{subject to}
& & 1/2 \cdot x_1 + x_2 = 1.
\end{aligned}
\end{equation*}

and a penalty on the form $\alpha(x) = (1/2 \cdot x_1 + x_2 - 1)^2$ is chosen so that the function to be minimized is:
$$
q(x,\mu) = e^{x_1} + x_1^2 + x_1 x_2 + \mu (1/2 \cdot x_1 + x_2 - 1)^2
$$

\subsubsection{Optimal points and function value}
The optimal points found is $ \mathbf{x} = (-1.278, 1.639) $ using the tolerance 1e-6. with the objective function minimum $-0.183$. Both DFP and BFGS methods converge to the same points with a difference $< 10^{-9}$. The series $\mu$ used was $(1, 10, 100, 1000, 10000, 100000, 1000000)$. It normally takes two outer iterations for the algorithm to stop.

The program execution converges to this feasible solution for a wide choice of $\mu$ in the penalty function. When $\mu$ is chosen small e.g $\mu = 0.1$ the solver runs into trouble, this is due to a minimum not existing. For large initial $\mu$ one can expect the conditioning of the problem cause problems for the solver, however, this problem is not very sensitive to the selection of $\mu$ and for DFP using a $\mu = 10^{18}$ and BFGS $\mu = 10^{16}$ still manages to converge. Selecting a larger initial $\mu$ also affects the number of outer iterations necessary for both methods.

\subsubsection{Typical output}
Typical outputs for solving using BFGS with only $\mu = 100$

\begin{verbatim}
Outer-Iteration 1
It 	x 	s.s 	f(x) 	|grad| 	l.s.i. 	lambda
1 	0.39 	0.89 	1.96 	2.57 	40 	3.98E-03
 	0.80
It 	x 	s.s 	f(x) 	|grad| 	l.s.i. 	lambda
2 	-1.28 	1.88 	-0.19 	0.38 	25 	7.33E-01
 	1.65
...
...
Outer-Iteration 3
It 	x 	s.s 	f(x) 	|grad| 	l.s.i. 	lambda
1 	-1.28 	0.00 	-0.19 	0.00 	0 	0.00E+00
 	1.65
It 	x 	s.s 	f(x) 	|grad| 	l.s.i. 	lambda
2 	-1.28 	0.00 	-0.19 	0.00 	0 	0.00E+00
 	1.65
Within tolerance, done!
Penalty/Barrier: 1.00E+02, function value: -1.87E-01
Found point y: (-1.283, 1.648)
Minimum value: -0.183
Penalty at y: 3.97E-13
Total iterations 3
\end{verbatim}
\subsubsection{Optimal points and function value}
\subsection{Exercise 9.5}
\subsubsection{Optimal points and function value}


%----------------------------------------------------------------------------------------
%	SECTION 3
%----------------------------------------------------------------------------------------
\section{Solver evaluation}
\subsection{Consistency in the program behaviour}
% Vet inte var vi ska l�gga detta
If the solution does not have a minimum there is a risk that the solver gets caught in an infinite loop. This can be solved by keeping track of the number of iterations performed in the line search or simply testing the function value in the limit to infinity, however we did not implement this. 

\subsection{Comparison between DFP and BFGS}

For problem 9.3 the only practical difference was the stable the algorithms used where to different large initial $\mu$.

%----------------------------------------------------------------------------------------



%%----------------------------------------------------------------------------------------
%%	BIBLIOGRAPHY
%%----------------------------------------------------------------------------------------
%
%\bibliographystyle{apalike}
%
%\bibliography{sample}
%
%%----------------------------------------------------------------------------------------
%
%
\end{document}